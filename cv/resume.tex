% LaTeX Curriculum Vitae Template
%
% Copyright (C) 2004-2008 Jason Blevins <jrblevin@sdf.lonestar.org>
% http://jblevins.org/projects/cv-template
%
% You may use use this document as a template to create your own CV
% and you may redistribute the source code freely. No attribution is
% required in any resulting documents. I do ask that you please leave
% this notice and the above URL in the source code if you choose to
% redistribute this file.
\documentclass[letterpaper]{article}
\usepackage{hyperref}
\usepackage{graphicx}
\usepackage{geometry}
\usepackage{bm}
% Uncomment the following lines to use the Palatino font.  Remove the
% [osf] bit if you don't like the old style figures.
%
% \usepackage[T1]{fontenc}
% \usepackage[osf]{mathpazo}
% Set your name here
\def\name{Siyu Xie}
% The following metadata will show up in the PDF properties
\hypersetup{
  colorlinks = true,
  urlcolor = black,
  pdfauthor = {\name},
  pdfkeywords = {economics empirical industrial organization econometrics
    microeconomics applied microeconomics},
  pdftitle = {\name: Curriculum Vitae},
  pdfsubject = {Curriculum Vitae},
  pdfpagemode = UseNone
}
\geometry{textheight=8.5in, textwidth=6in}
% Customize page headers
\pagestyle{myheadings}
\markright{\name}
\thispagestyle{empty}
% Customize section headings
\usepackage{sectsty}
\subsectionfont{\rmfamily\mdseries\itshape\large}
% Don't indent paragraphs.
\setlength\parindent{0em}
% Make lists without bullets
\renewenvironment{itemize}{
  \begin{list}{}{
    \setlength{\leftmargin}{1em}
  }
}{
  \end{list}
}
\begin{document}
\centerline{\huge\bf \name}
\vspace{0.25in}
\begin{minipage}[t]{0.8\textwidth}
Department of Electrical and Computer Engineering (ECE)\\
Wayne State University (WSU)\\
5050 Anthony Wayne Drive, Detroit, Michigan 48202, USA\\
%\emph{Sex}: Female\\
%\emph{Birth}: May $11^{th}$ 1991\\
Email: {\tt syxie@wayne.edu}\\
%\emph{Homepage}: https://xiesiyu7.github.io/\\
\end{minipage}

\section*{Education}
2018.04-present~~~~Postdoc in Wayne State University, Detroit, MI 48202, USA
\begin{itemize}
    \item ~~~~~~~~~~~~~~~~~~~~~\textit{Supervisor}: Prof. Le Yi Wang (IEEE Fellow) and Prof. Masoud Nazari
    \end{itemize}
2018.09-2019.03~~~~Visiting scholar in Academy of Mathematics and Systems Science (AMSS)
\begin{itemize}
    \item ~~~~~~~~~~~~~~~~~~~~~~Chinese Academy of Sciences (CAS)
    \end{itemize}
2013.09-2018.06~~~~Key Laboratory of Systems and Control
    \begin{itemize}
    \item ~~~~~~~~~~~~~~~~~~~~~~AMSS, CAS, Beijing, China
    \item ~~~~~~~~~~~~~~~~~~~~~~Ph.D degree in control theory
    \item ~~~~~~~~~~~~~~~~~~~~~~\textit{Supervisor}: Prof. Lei Guo (Academician of Chinese Academy of Sciences)
    \end{itemize}
2009.09-2013.06~~~~~Department of Mathematics and Systems Science
    \begin{itemize}
    \item ~~~~~~~~~~~~~~~~~~~~~~~Beihang University, Beijing, China
    \item ~~~~~~~~~~~~~~~~~~~~~~~Bachelor degree in information and computing science (systems control)
    \end{itemize}

\section*{Research Interests}


Distributed optimization for power systems, distributed adaptive filters, adaptive signal processing, distributed control theory, networked systems, machine learning, compressive sensing


\section*{Skills}
\begin{itemize}
\item $\bm{\cdot}$ MATLAB, LATEX, Python
\item $\bm{\cdot}$ (adaptive) filtering algorithms, estimation of unknown (time-varying) parameters, system identification, adaptive control, distributed adaptive filters, optimization theory, multi-agent system, some general machine learning algorithms and signal processing theory, probability theory and mathematical statistics
\end{itemize}


\section*{Honors}
\begin{itemize}
\item $\bm{\cdot}$ President Scholarship of Chinese Academy of Sciences, 2018.
\item $\bm{\cdot}$ China National Scholarship for graduate students, 2017. (top 15 among 600 graduate students)
\item $\bm{\cdot}$ Excellent Student of University of Chinese Academy of Sciences, 2016.
\item $\bm{\cdot}$ IEEE Control Systems Society Beijing Chapter Youth Author Award, 2015.
\item $\bm{\cdot}$ Outstanding Student Scholarship of CAS, 2013.
\end{itemize}


\section*{Publications}
\begin{itemize}
\item $\bm{\cdot}$ D. Gan, S. Y. Xie, and Z. X. Liu, \emph{Stability of the Distributed Kalman Filter With General 
Random Coefficients}, accepted by \textbf{SCIENCE CHINA Information Sciences}, 2020. (Regular paper)
\item $\bm{\cdot}$ S. Y. Xie and L. Guo, \emph{Analysis of Compressed Distributed Adaptive Filters}, \textbf{Automatica}, vol. 112, February 2020. (Regular paper)
\item $\bm{\cdot}$ P. Zhao, X. Q. Wang, S. Y. Xie, L. Guo and Z. H. Zhou, \emph{Distribution-free One-pass Learning}, \textbf{IEEE Transactions on Knowledge and Data Engineering}, August 2019. (Regular paper)
\item $\bm{\cdot}$ S. Y. Xie and L. Guo, \emph{Analysis of Distributed Adaptive Filters Based on Diffusion Strategies over Sensor Networks}, \textbf{IEEE Transactions on Automatic Control}, vol. 63, no. 11, pp. 3643--3658, November 2018. (Regular paper)
\item $\bm{\cdot}$ S. Y. Xie and L. Guo, \emph{A Necessary and Sufficient Condition for Stability of LMS-based Consensus Adaptive Filters}, \textbf{Automatica}, vol. 93, pp. 12-19, July 2018. (Regular paper)
\item $\bm{\cdot}$ S. Y. Xie and L. Guo, \emph{Analysis of NLMS-based Consensus Adaptive Filters under a General Information Condition}, \textbf{SIAM Journal on Control and Optimization} , vol. 56, no. 5, pp. 3404-3431, January 2018. (Regular paper)
\item $\bm{\cdot}$ S. Y. Xie and L. Guo, \emph{Exponential Stability of LMS-Based Distributed Adaptive Filters}, Proc.of the 20th World Congress of the International Federation of Automatic Control (\textbf{IFAC}), Toulouse, France, July 9-14, 2017, pp. 13764-13769.	
\item $\bm{\cdot}$ S. Y. Xie and L. Guo, \emph{Compressive Distributed Adaptive Filtering}, 35th Chinese Control Conference (\textbf{CCC}), Chengdu, China, July 2016, pp. 5229-5234.
\item $\bm{\cdot}$ S. Y. Xie and L. Guo, \emph{Stability of Distributed LMS under Cooperative Stochastic Excitation}, 34th Chinese Control Conference (\textbf{CCC}), Hangzhou, China, July 2015, pp. 7445-7450.
\end{itemize}

\section*{Papers in Preparation}
\begin{itemize}
\item $\bm{\cdot}$ S. Y. Xie, S. Liang, L. Y. Wang, G. Yin and W. Chen, \emph{Stochastic Adaptive Optimization with Dithers}, submitted to \textbf{IEEE Transactions on Automatic Control}, 2020. (Regular paper)
\item $\bm{\cdot}$ S. Y. Xie, M. H. Nazari, L. Y. Wang, G. Yin and W. Chen, \emph{Impact of Stochastic Generation/Load Variations on
Distributed Optimal Energy Management in DC
Microgrids}, submitted to \textbf{IEEE Transactions on Intelligent Transportation Systems}, 2020. (Regular paper)
\item $\bm{\cdot}$ M. H. Nazari, S. Y. Xie, L. Y. Wang, G. Yin and W. Chen, \emph{Impact of Communication Packet Delivery Ratio on Reliability and Resilience in Optimal Power Management of DC Microgrids}, submitted to \textbf{IEEE Transactions on Smart Grid}, 2020. (Regular paper)
\item $\bm{\cdot}$ A. S. Matveev, M. Almodarresi, R.
Ortega, A. Pyrkin and S. Y. Xie, \emph{Diffusion-based Distributed Parameter Estimation Through Directed Graphs with Switching Topology: Application of Dynamic Regressor Extension and Mixing}, submitted to \textbf{IEEE Transactions on Automatic Control}, 2020. (Regular paper)
\item $\bm{\cdot}$ S. Y. Xie, Y. Q. Zhang and L. Guo, \emph{Learning and Prediction Theory of Distributed Least Squares}, submitted to \textbf{IEEE Transactions on Automatic Control}, 2019. (Regular paper)
\end{itemize}



%\section*{Research Statement}
%I have studied the stability and performance analyses of the distributed adaptive filtering algorithms under non-independent and non-stationary signal assumptions, and I plan to further study some other distributed filtering and learning algorithms in the future. For the first time, these results have laid a theoretical foundation for the distributed adaptive filters in stochastic feedback systems, which is significant in the network era.

%\section*{Main Research Contributions}

%\begin{itemize}
%\item $\bm{\cdot}$ For the theoretical analysis of the distributed adaptive filters, most of the existing theory in the literature require stringent conditions that the signals are independent and stationary, while our work does not.

%\item $\bm{\cdot}$ The weakest possible information condition used in our work can be used to demonstrate that the distributed adaptive filters can work well even if any individual filter cannot due to lack of necessary information.

%\item $\bm{\cdot}$ For the first time, these results have laid a theoretical foundation for the distributed adaptive filters in stochastic feedback systems, which is significant in the network era.

%\item $\bm{\cdot}$ Our work makes it possible for further investigations on related problems concerning the combination of control and communication.
%\end{itemize}

%\newpage

%\section*{Presentations and Invited Plenary Lectures}
%\begin{itemize}
%\item $\bm{\cdot}$ 34th Chinese Control Conference, Hangzhou, China, July 2015,\\
%lecture titled ``Stability of Distributed LMS under Cooperative Stochastic Excitation''.
%\item $\bm{\cdot}$ 35th Chinese Control Conference, Chengdu, China, July 2016,\\
%lecture titled ``Compressive Distributed Adaptive Filtering''.
%\item $\bm{\cdot}$ 20th IFAC World Congress, Toulouse, France, July 2017,\\
%invited lecture titled ``Exponential Stability of LMS-Based Distributed Adaptive Filters''.
%\item $\bm{\cdot}$ L. Guo and S. Y. Xie, ``Distributed Adaptive Filtering,'' Plenary Lecture at the 6th IFAC Workshop on Distributed Estimation and Control in Networked Systems (NecSys¡¯2016), Tokyo, Japan, September 2016.    (http://www.necsys2016.ctrl.titech.ac.jp/slides/lguo.pdf)
%\item $\bm{\cdot}$ L. Guo and S. Y. Xie, ''Distributed Adaptive Filtering Over Sensor Networks,'' Plenary Lecture at the 12th Chinese Conference on Complex Networks (CCCN 2016), Taiyuan, Shanxi, October 2016.    (http://cccn2016.i.dian.in/pages/yao-qing-bao-gao)
%\end{itemize}

%\section*{Major Courses Taken}
%Classical and modern control theories, optimal control theory, optimization theory, time-varying stochastic system, system identification, signal processing, statistical machine learning theory, game theory, operational research, graph theory, probability theory and mathematical statistics

%\section*{Language}
%\begin{itemize}
%\item $\bm{\cdot}$ Fluent in English, speaking and writing
%\item $\bm{\cdot}$ Chinese (Native)
%\end{itemize}



%\section*{Referrers}

%\begin{itemize}
%\item 1. Professor Lei Guo\\
%Academy of Mathematics and Systems Science, Chinese Academy of Sciences, Beijing, China.\\
%\emph{Email}: {\tt lguo@amss.ac.cn}
%\item 2. Professor Jifeng Zhang\\
%Academy of Mathematics and Systems Science, Chinese Academy of Sciences, Beijing, China.\\
%\emph{Email}: {\tt jif@iss.ac.cn }
%\item 3. Professor Leyi Wang\\
%Department of Electrical and Computer Engineering, Wayne State University, Detroit, MI, USA.\\
%\emph{Email}: {\tt lywang@wayne.edu}
%\item 4. Professor Miroslav Krstic\\
%Department of Mechanical and Aerospace Engineering, University of California, San Diego, CA, USA.\\
%\emph{Email}: {\tt krstic@ucsd.edu}

%\end{itemize}


\end{document}
